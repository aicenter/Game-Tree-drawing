%%%%%%%%%%%%%% Packages used for drawing the trees %%%%%%%%%%%%%%%%%%%%%%%%%
\usepackage{graphicx}

% for coloring figures; load before tikz to use dvipsnames
% if you get option clash for xcolor, replace 'Green' with 'cyan' in color defs
\usepackage{xcolor}
%\usepackage{tikz} % for drawing (GT graphs in particular)
\usepackage{forest} % should be better than the standard tikz trees
	\usetikzlibrary{shapes} % for triangle shaped nodes
	\usetikzlibrary{calc} % for calculating coordinates
\usepackage{xparse}	% for defining macros with optional argumentss

%%%%%%%%%%%%%%% Node style definitions %%%%%%%%%%%%%%%%%%%%%%%%%%%%%%%%%%%%%%%%%%%%%%
% node sizes setup
\newlength{\nodesize}
% this is meant to be adjustable, but doing so messes up level distances, and requires adjusting l sep for chance and terminal nodes
\setlength{\nodesize}{2.5em}
% Player colors
\colorlet{chance_color}{black}
\colorlet{pl0_color}{chance_color}
\colorlet{chance_text}{white}
\colorlet{pl1_color}{magenta!50}
\colorlet{pl2_color}{cyan!50}
\colorlet{pl0_infoset_color}{pl0_color}
\colorlet{pl1_infoset_color}{magenta!75}
\colorlet{pl2_infoset_color}{cyan!75}
% player 1 and 2, chance, terminal nodes
\forestset{
	basenode/.style = {draw,
		inner sep = 0,
		outer ysep = 0,
		minimum size = \nodesize,
		anchor = north
	},
	playernode/.style={basenode, 
		shape = regular polygon,
		regular polygon sides = 3,
	},
	pl1/.style={playernode, fill=pl1_color},
	pl2/.style={playernode, fill=pl2_color, shape border rotate=180},
	chance/.style = {basenode,
		fill=pl0_color, text=chance_text,
		circle,
		minimum size=0.75*\nodesize,
%		child anchor=center,
		l sep=0.4765\nodesize,
	},
	terminal/.style = {basenode,
		shape = regular polygon,
		regular polygon sides = 4,
		l sep=0.47\nodesize,
		minimum size = 1.07\nodesize
	}
}
% information sets
\tikzset{
	partition/.style = {
		draw,
%		inner xsep = 0.1\nodesize,
%		inner ysep = 0.1\nodesize,
		rounded corners = 5,
		inner sep=0.1\nodesize,
	},
	infoset/.style = {
		partition,
		draw=pl1_infoset_color,
	},
	augmented/.style = {
		dashed
	},
	opponent/.style = {
		draw=pl2_infoset_color,
		% bigger separation than pl1, to prevent overlapping with pl1 infosets
		inner sep=0.225\nodesize,
	},
	% triangles are not symmetric, the ``classical'' infosets shift the rectangles
	% to make the spaces above and below nodes look more similar
	pl1_cl_infoset/.style = {infoset, yshift=-0.035\nodesize},
	pl2_cl_infoset/.style = {infoset,
		opponent,
		inner sep=0.1\nodesize,
		yshift=0.035\nodesize
	},
	public_state/.style = {
		partition,
		draw=pl0_infoset_color,
		inner sep=0.125\nodesize,
		rounded corners = 7,
	},
}
% infosets connected by lines
\tikzset{
	line_infoset/.style = {
		draw=pl1_infoset_color,
		dashed
	},
	opp_line_infoset/.style = {
		draw=pl2_infoset_color,
		dashed
	}
}
% a shortcut for listing all corners of a triangle-node
\newcommand{\corners}[1]{#1.corner 1)(#1.corner 2)(#1.corner 3}
% measure the angle between node A and B (without '()' brackets around)
% and save it into `\angle`
\newcommand{\measureAngle}[2]{
	\node(dummy#1)[draw=none] at (#1) {};
	\node(dummy#2)[draw=none] at (#2) {};
	\pgfmathanglebetweenpoints
		{\pgfpointanchor{dummy#1}{center}}
		{\pgfpointanchor{dummy#2}{center}}
	\edef\angle{\pgfmathresult}
}
%%%%%%%%%%%%%%%%%%%%%%%%%%%%%%%
%%%%%%%%% just some nonsense to be able to have infosets in the background
\pgfdeclarelayer{bg}    % declare background layer
\pgfsetlayers{bg,main}  % set the order of the layers (main is the standard layer)
%%%%%%%%%%%%%%%%%%%%%%%%%%%%%%%%%